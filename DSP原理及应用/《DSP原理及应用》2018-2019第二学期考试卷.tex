\documentclass{ctexart}
\usepackage{amsmath,amsthm,siunitx}
\usepackage{ulem}
\usepackage[a4paper,margin=1in]{geometry}
\theoremstyle{definition}
\newtheorem{ti}{}[section]
\def\theti{\arabic{ti}}
\def\hua{\uline{\hspace*{4pc}}}

\title{《DSP 原理及应用》2018-2019第二学期考试卷\thanks{试卷编号:1819020616B}}
\author{秦淑雅}
\begin{document}
	\pagestyle{plain}
	\maketitle
	\section{填空题($40$ 分)}
	\begin{ti}
		配置 IO 口工作于外设功能或数字 IO 功能的寄存器是\hua,复位时所有 GPIO 配制成\hua 功能状态;配置 IO 口方向的寄存器是\hua,复位时所有 GPIO 为\hua(输入/输出)状态。
	\end{ti}

	\begin{ti}
		X2812xDSP 的中断向量表地址由\hua、\hua、\hua、\hua 信号控制。
	\end{ti}

	\begin{ti}
		复位时\hua 引脚被采样为低电平,锁相环被禁止;\hua 引脚是功率保护引脚,下降沿引发功率驱动保护中断将 EVA 的 PWM 输出引脚置为高阻态。
	\end{ti}

	\begin{ti}
		T1 的 TMS320X281X 系列 DSP 为了保护关键寄存器,在对这些特殊寄存器改写之前要执行汇编指令“asm(“\hua”)”以置位 ST1 的 D6 位,设置寄存器执行之后要执行“asm(“\hua”)”以清除 ST1 的 D6 位;这些需要保护的特殊功能寄存器是\hua、\hua、\hua、\hua、\hua、\hua、\hua。
	\end{ti}

	\begin{ti}
		通用定时器的比较单元产生高有效的 PWM 对称波形时占空比公式为\hua。
	\end{ti}

	\begin{ti}
		定期“喂狗”实际就是周期性向\hua 寄存器写入\hua+\hua。
	\end{ti}

	\begin{ti}
		记录引脚电平跳变时刻可以用事件管理器的\hua 单元。
	\end{ti}

	\begin{ti}
		语句“\hua”将 ADC 的寄存器变量 AdcRegs 定位到 AdcRegsFile 段中。
	\end{ti}

	\begin{ti}
		A/D 初始化函数文件名为\hua;CPU 定时器配置函数为\hua。
	\end{ti}

	\begin{ti}
		可执行文件后缀是*.\hua,链接命令文件后缀是*.\hua。
	\end{ti}

	\begin{ti}
		定时器比较匹配事件时 TxPWM/TxCMP 引脚由低电平跳变到高电平则该引脚的输出极性模式为\hua。
	\end{ti}

	\begin{ti}
		使能捕获单元 1 和 2,需要写指令 EvaRegs.\hua.\hua.CAP12PN=1。
	\end{ti}

	\begin{ti}
		TMS320X2812 扩展片外数据存储器选择 XINTF6 区,起始地址是\hua,存储器片选信号与 DSP 的\hua 引脚相连接。
	\end{ti}

	\begin{ti}
		为使外设中断被响应后 PIE 控制器能响应同组的其他中断要对\hua 的相关位进行手动复位,即对相应位写\hua。
	\end{ti}

	\begin{ti}
		\hua 文件中有一个函数\uline{\hspace*{4pc}(函数名)}实现对外设中断扩展模块 PIE 控制寄存器进行初始化。
	\end{ti}

	\begin{ti}
		INT1.5 是\hua 中断。
	\end{ti}

	\begin{ti}
		设置全比较单元引脚输出极性的寄存器为\hua。
	\end{ti}

	\section{简答题($10 \times 3$)分}
	\begin{ti}
		使捕获单元工作需要进行什么设置?详细说明 CAPFIFOA 的 D9D8 位作用。
	\end{ti}

	\begin{ti}
		已知使用的晶体振荡器频率,需要设置哪些寄存器,确定通用定时器的时钟基准(定时器计数一个节拍的时钟周期)。
	\end{ti}

	\begin{ti}
		使通用定时器 T1、T2 同步的设置步骤。
	\end{ti}

	\section{编程题($10 + 20$ 分)}
	\begin{ti}
		外部晶振频率为 \SI{30}{MHz},希望得到 SYSCLKOBT 为 \SI{150}{MHz},高速外设时钟为 \SI{75}{MHz},低速外设时钟为 \SI{37.5}{MHz},禁止看门狗使用 EVA、ADC 以及 SP1 外设,写出系统初始化程序。
		\begin{verbatim}
			#include"_________"
			/*功能:对 F2812 系统控制寄存器初始化
			入口参数:无
			出口参数:无*/
		\end{verbatim}
	\end{ti}

	\begin{ti}
		使用通用定时器 1 每隔 \SI{1}{ms} 发生一次中断,在 GPIOB0 引脚上产生周期 \SI{0.8}{s} 的方波(400 次中断电平跳变一次);在高速外设时钟为 \SI{75}{MHz} 的设定下,分模块写出通用定时器 1 的初始化设置函数,系统初始化调用和中断初始化设置,引脚初始化函数,中断服务函数。
	\end{ti}
\end{document}